\section{Related Work}\label{sec:relatedwork}
\subsection{Photometric Stereo Under Various Lighting}
Woodham's original paper~\cite{woodham1979} describes photometric stereo under
a single directional light source per image, such as that of an infinitely
distant point source. As this naturally restricts the utility of the method to
carefully calibrated indoor environments, much research since then has been
done in relaxing this requirement.

Clark et al.~\cite{clark2006} show that nearby planar light sources, such as
LCD computer monitors~\cite{clark2010}, are effective approximations of distant
point sources and can therefore be used in conjunction with the original
algorithm. Deviating from the original assumptions, several groups including
Xie et al.~\cite{xie} produce methods for photometric stereo using
\emph{nearby} point sources, characterizing it as a distortion of the original
problem that can be solved by iterating mesh-deformations until they converge.
Gan and Thorm\"{a}hlen~\cite{gan} take a similar approach in providing an
iterative algorithm for the same problem with area lights. The goal of the
latter two papers is the same: to apply a little more computation in performing
photometric stereo in exchange for a more practical lighting setup.

More generally, Basri and Jacobs~\cite{basri2001-12} provide a theoretical
framework for photometric stereo under \emph{any} combination of distant
isotropic lights, building on their previous finding~\cite{basri2001-07} that
the reflectance function of any Lambertian surface under such lighting is
well-approximated by low-order spherical harmonics. This extends Woodham et
al.'s 1991 report~\cite{woodham1991} that the light sources used in photometric
stereo need not be known \textit{a priori}. Finally, in~\cite{schechner}
Schechner et al.\ show specifically that the traditional matrix formulation of
photometric stereo can easily be adapted to fit two light sources per capture
instead of one. These results demonstrate the feasibility, in general, of
performing photometric stereo under two or more unknown light sources. This is
a slightly less constrained problem than our own, since the light from our
primary source and its mirror image are coupled in any given frame.
\subsection{Scene Reconstruction With Mirrors}
Beyond photometric stereo, a variety of techniques have been developed to adapt
to and use mirrors in scene reconstruction. Noting that mirrors and glass
surfaces, while traditionally difficult for visual approaches to depth sensing,
are particularly suited to acoustic methods, Zhang et al.~\cite{zhang}
demonstrate a sensor fusion method with a first-generation Kinect camera (a
structured-light device) and an ultrasonic range sensor that is capable of
producing a dense depth map over a variety of surfaces.

In some cases, mirrors are explicitly incorporated into the lighting setup to
improve the reconstruction technique. Lanman et al.~\cite{lanman} take
advantage of the complex interactions between structured light and mirrors,
which~\cite{zhang} avoids using fused acoustic data, to make 3D scanning faster
and allow a single camera to scan many sides of an object at once. A Fresnel
lens is used to project the light structure orthographically. Ahn et
al.~\cite{ahn} generalize this approach, removing the requirement that the
mirrors be aligned along an axis by presenting the unravelling of the multiple
image views as a labelling problem, allowing the system to view an object being
scanned from even more angles and resolving otherwise inherent problems with
inset areas. Finally, Xu et al.~\cite{xu} completes the mirrors into a light
trap, inferring from a ToF camera and the known shape of the trap the complete
light path of a ray ultimately reflected by the diffuse object inside. Note
that although these methods are effective at utilizing numerous views of an
object from multiple mirrors, they do so while restricted to objects small
enough to fit in their setup and with the requirement that the operator be
allowed to arrange a set of mirrors with high precision, two limitations the
method in this paper seeks to avoid.

It's worth noting that because mirrors are frequently a point of failure in
na\"{\i}ve approaches to certain reconstruction algorithms, some groups take
the explicit first step of \emph{detecting} a mirror in the scene. Yang et
al.~\cite{yang} take a deep-learning view of the problem, introducing both
MirrorNet and a labelled dataset of images containing mirrors for training it.
One key insight they use is the existence of abrupt contextual contrast inside
and outside of the mirror image, which helps the network infer and segment the
mirror boundaries. This line of work is taken further by Lin et al.\ in the
same group~\cite{lin}, which takes advantage of the fact that the content
inside of the mirror is a reflection of something outside of it to not only
improve on detection performance but also implicitly correlate the mirror image
with its original.  Although this project does not make use of mirror detection
algorithms, the existence of these methods indicates possible future research
in end-to-end detection, localization, and adaptation of mirrors in scenes for
photometric stereo.
