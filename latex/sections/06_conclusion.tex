\section{Conclusion}\label{sec:conclusion}
Photometric stereo is no doubt an essential component of the pantheon of
techniques for recovering 3D scene properties from purely visual information.
Making it work in difficult scenarios---scenes whose properties do not match
the original assumptions of light and surface properties---is an important
ongoing area of research.

In this paper, we looked at the applicability and implications of photometric
stereo in scenes containing a stationary perfect mirror, providing a
characterization of the problem and proposing a set of techniques for resolving
the inherent ambiguity that arises when multiple light sources---in this case,
the original light source and its mirror image---are present in a single frame.
Unfortunately, it appears that the basic linear problem of photometric stereo
is ill posed in the presence of such a mirror. We conclude with two statements:
that any setup for directional-light photometric stereo with mirrors requires
the mirrors to either change in position between frames or have a reflectivity
significantly less than 1, and that the task of photometric stereo with mirrors
in general reduces to a light-labelling problem of the sort described by
Schechner et al.\ in~\cite{schechner}, a problem for which the most efficient
solution still appears to be their ShadowCuts technique.
\subsection{Future Work}
Although the particular setup in our project was not condusive to a solution,
there are still plenty of questions worth pursuing in this space. The most
obvious is to continue investigating scenes for which photometric stereo
methods \emph{are} appropriate. As discussed above, these would be scenes in
which the mirror either changes in position from frame to frame or has a
reflectivity value of less than 1.

We might expect the former to have less utility due to its implications. If a
mirror is an inherent part of the scene, then it would not be likely (nor
desirable, as it implies scene motion) for the mirror to change position
between captures. If a mirror is part of the capture apparatus, then there
isn't a clear benefit of moving a mirror around between frames as opposed to
simply replacing it with another light source, which leads to a similar problem
characterization anyway. In fact, using a moving mirror instead of another
light souorce carries the additional drawback that a mirror can't simply be
switched on and off between frames as light sources are in~\cite{schechner}.

On the other hand, a mirror with lower reflectivity may be a simple solution to
this problem with no added complexity. The main drawback would be the lower
level of illumination on the object. It would be worth conducting further
experiments to see what values of reflectivity induce a good balance between
avoiding light vector coplanarity and providing enough illumination.

A second line of proposed future work involves modifying the characteristics of
the direct or reflected light in order to make it easier to pick apart the
direct and reflected illumination components on each pixel. Three approaches
are proposed:
\begin{enumerate}
  \item Apply a color filter to the mirror, so that it is only reflective for a
  subset of the incident wavelengths (from the direct illumination source).
  This can be combined with the lower-reflectivity mirror proposal above to
  reduce the number of custom components in the capture setup. A single mirror
  that reflects only red light with known non-unit reflectivity, for example,
  may work.
  \item Direct the incoming light at a glancing angle to the mirror surface.
  Because mirrors tend to induce polarization when reflecting light at shallow
  angles, the degree to which the light received by the sensor is polarized can
  be used to distinguish between direct and mirrored light sources.
  \item Use circularly polarized light in the direct source. Because mirrors
  reverse the direction of polarization, this can also be used as a way to
  distinguish between direct and mirrored light sources.
\end{enumerate}
Each of these three methods, by changing the characteristics of the mirrored
light with respect to the direct, may actually obviate the need for a
source-labelling algorithm as described in Section~\ref{sec:implementation}, as
the captured light would be more analogous to that of multi-band photometric
stereo, in which as few as a single image with multiple light channels can be
used to recover surface normals.

