\begin{abstract}
  We investigate the utilization of mirrors in a scene as a way to augment the
  recovery of scene geometry through traditional photometric stereo techniques.
  Light from a known primary source reflected on the mirror before encountering
  the scene is modelled as a second light source, reducing the challenge of
  photometric stereo with mirrors to that of photometric stereo with multiple
  light sources per image. Such a method would transform scene mirrors from a
  source of error in depth estimation to a source of additional information
  about the shape of diffuse objects, enabling the reconstruction of otherwise
  obstructed areas, such as self-occluded areas and sides of objects facing
  away from the camera. However, we find that the geometric implications of a
  single perfectly reflective stationary mirror render existing techniques to
  determine the combination of lights incident at a scene point impossible to
  resolve. Further work is needed to determine the feasibility of following
  this method with either moving (frame-dependent) or partly-absorptive mirror
  surfaces.
\end{abstract}

