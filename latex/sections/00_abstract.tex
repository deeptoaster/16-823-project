\begin{abstract}
  We investigate the utilization of existing mirrors in a scene as a way to augment the recovery of scene geometry through traditional photometric stereo techniques. The image received from an image plane is modelled as a second camera viewing target objects from a different angle, and light from a known primary source reflected on the object is modelled as an additional light source. Such a method transforms scene mirrors from a source of error in depth estimation to a source of additional information about the shape of diffuse objects, enabling the reconstruction of otherwise obstructed areas, such as self-occluded areas and sides of objects facing away from the camera. Furthermore, we use the known geometry of the mirror in conjunction with a Lambertian assumption of target object surfaces to estimate the position of a primary light source, demonstrating the combined recovery of scene lighting and scene geometry from a single capture requiring only the position of a mirror.
\end{abstract}

